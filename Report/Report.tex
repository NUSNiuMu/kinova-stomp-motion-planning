% !TEX program = xelatex
\documentclass[12pt,a4paper]{ctexart}

% 基本宏包
\usepackage{amsmath,amssymb}
\usepackage{graphicx}
\usepackage{booktabs}
\usepackage{hyperref}
\usepackage{caption}
\usepackage{subcaption}
\usepackage{xcolor}
\usepackage{listings}
\usepackage{geometry}
\geometry{left=2.5cm,right=2.5cm,top=2.5cm,bottom=2.5cm}

\hypersetup{
  colorlinks=true,
  linkcolor=blue,
  citecolor=blue,
  urlcolor=blue
}

\graphicspath{{figures/}}

% 代码高亮(主要用于 MATLAB 片段)
\lstdefinestyle{mystyle}{
  language=Matlab,
  basicstyle=\ttfamily\small,
  keywordstyle=\color{blue},
  commentstyle=\color{teal!70!black},
  stringstyle=\color{orange!70!black},
  numbers=left,
  numberstyle=\tiny,numbersep=8pt,
  breaklines=true,
  frame=single,
  framerule=0.3pt,
  columns=fullflexible
}
\lstset{style=mystyle}

% 封面信息(按 README 与课程信息)
\title{\textbf{EE5112 人机交互 — Project 2\\基于 STOMP 的 Kinova 机械臂轨迹规划}}
\author{\textbf{Team Members:}~Wu Zining\quad Niu Mu\quad Zhao Jinqiu}
\date{AY 2025/2026\\National University of Singapore}

\begin{document}

\maketitle

\begin{center}
\vspace{-6pt}
\textit{Lecturer: Dr. Lin Zhao (School of ECE, NUS)}\\
\textit{Codebase: kinova-stomp-motion-planning}
\end{center}

\vspace{8pt}

\begin{abstract}
本文围绕 EE5112 Project 2 要求,完成对基于 STOMP (Stochastic Trajectory Optimization for Motion Planning) 的轨迹规划实现与评估。报告重点详述:\textbf{Task 1} — 补全示例代码、在原始障碍设置下实现 Kinova Gen3 机械臂的无碰撞路径规划与可视化;\textbf{Task 3} — 基于指数积 (Product of Exponentials, PoE) 公式实现正向运动学以替代内置 \verb|getTransform()|,并与 STOMP 流水线衔接。\textbf{Task 2/4/5} 按要求仅保留标题占位,无正文。我们给出算法原理、实现要点、关键参数、实验设置与结果分析,并附参考文献以支撑方法选择与实现细节。
\end{abstract}

\textbf{关键词:}STOMP,运动规划,PoE,正向运动学,Kinova Gen3,避障

\tableofcontents

\section{任务一:基于 STOMP 的 Kinova 机械臂无碰撞路径规划}

\subsection{任务目标与待补全模块}
本任务要求完善给定的不完整示例代码,使 MATLAB Live Script \verb|KINOVA_STOMP_Path_Planning.mlx| 能够在原始障碍场景下,为 Kinova Gen3 机械臂规划一条从初始配置到目标末端姿态的\textbf{无碰撞、平滑}轨迹,并生成可视化动画。项目明确指出需要补全以下五个核心函数模块:

\begin{itemize}
  \item \verb|helperSTOMP.m| — STOMP 主循环与迭代控制
  \item \verb|updateJointsWorldPosition.m| — 正向运动学计算(Task 3 用 PoE 替换)
  \item \verb|stompDTheta.m| — 梯度估计(加权噪声求和)
  \item \verb|stompSamples.m| — 轨迹采样(多元高斯扰动生成)
  \item \verb|stompObstacleCost.m| — 障碍代价计算(基于符号距离场)
\end{itemize}

\subsection{STOMP 算法原理}

STOMP(Stochastic Trajectory Optimization for Motion Planning)\cite{Kalakrishnan2011STOMP} 是一种基于随机采样的轨迹优化方法,其核心思想是:在给定初始轨迹的基础上,通过\textbf{加噪声采样}、\textbf{代价评估}、\textbf{加权更新}三个步骤迭代优化轨迹,无需显式计算梯度,因此对非光滑、不可导的代价函数(如碰撞惩罚)具有良好的鲁棒性。

\subsubsection{算法流程}

设轨迹由 $T$ 个离散时间步的关节配置 $\{\theta_t\}_{t=1}^T$ 描述($\theta_t \in \mathbb{R}^n$),其中 $\theta_1$ 和 $\theta_T$ 为固定的起点与终点。算法迭代过程如下:

\paragraph{Step 1:采样}
对每个内部时间步 $t \in \{2,\dots,T-1\}$,生成 $K$ 条带噪声的采样轨迹:
\[
\tilde{\theta}_t^{(k)} = \theta_t + \varepsilon_t^{(k)}, \quad \varepsilon_t^{(k)} \sim \mathcal{N}(0, \Sigma),\; k=1,\dots,K
\]
其中协方差矩阵 $\Sigma$ 通常取为平滑矩阵 $R$ 的逆(归一化后),以鼓励轨迹在时间上的连续性。

\paragraph{Step 2:代价评估}
对每条采样轨迹 $k$,计算其总代价:
\[
C^{(k)} = \sum_{t=1}^T c(\tilde{\theta}_t^{(k)}) + \frac{1}{2}\tilde{\theta}^{(k)\top} R \tilde{\theta}^{(k)}
\]
其中 $c(\theta_t)$ 为障碍代价,$R$ 为二阶差分平滑矩阵。

\paragraph{Step 3:概率加权}
将代价转换为概率权重(采用 Boltzmann 分布):
\[
w^{(k)} = \frac{\exp(-\eta^{-1} C^{(k)})}{\sum_{j=1}^K \exp(-\eta^{-1} C^{(j)})}
\]
其中 $\eta$ 为温度参数,控制代价对概率的敏感度。

\paragraph{Step 4:梯度估计与更新}
计算加权噪声的期望作为更新方向:
\[
\Delta\theta_t = \sum_{k=1}^K w^{(k)} \varepsilon_t^{(k)}
\]
应用平滑后的更新:
\[
\theta_t \leftarrow \theta_t + M \Delta\theta_t
\]
其中 $M$ 为平滑矩阵,通常由 $R$ 的逆归一化得到。

\subsection{代价函数设计}

我们的代价函数由三部分组成:

\subsubsection{障碍代价 $c_{\text{obs}}$}
采用基于\textbf{符号欧氏距离场}(Signed Euclidean Distance Transform, sEDT)的指数惩罚\cite{Khatib1986Potential}。对机器人每一连杆用一系列球体近似(球心由 \verb|stompRobotSphere.m| 生成),计算每个球心到最近障碍的距离 $d_i$:
\[
c_{\text{obs}} = \sum_{i} \max\left(0, \exp\left(\alpha(\delta_i)^2\right) - 1\right), \quad \delta_i = d_{\text{safe}} - d_i
\]
其中 $d_{\text{safe}}=0.1$m 为安全裕度,$\alpha=200$ 为惩罚强度。仅当 $d_i < d_{\text{safe}}$ 时施加惩罚。

\subsubsection{平滑代价 $c_{\text{smooth}}$}
采用二阶有限差分矩阵 $R$ 惩罚加速度:
\[
c_{\text{smooth}} = \frac{1}{2}\theta^{\top} R \theta, \quad R = A^{\top}A
\]
其中 $A$ 为离散二阶差分算子。该项确保轨迹在关节空间的平滑性,避免抖动。

\subsubsection{约束代价 $c_{\text{constraint}}$}
预留接口用于添加末端姿态约束(Task 5)。当前实现中设为零:
\[
c_{\text{constraint}}(t) = 0
\]

\subsection{关键实现模块}

\subsubsection{\texttt{helperSTOMP.m} — 主循环}
实现完整的 STOMP 迭代流程,包括:
\begin{itemize}
  \item 轨迹初始化(线性插值)
  \item 平滑矩阵预计算($R$、$R^{-1}$、$M$)
  \item 收敛判定(代价变化小于阈值或达到最大迭代次数 50)
  \item 碰撞检测(使用 MATLAB \verb|checkCollision|)
  \item 动画生成(可选开关 \verb|enableVideo| 与 \verb|enableVideoTraining|)
\end{itemize}

关键参数设置:
\begin{itemize}
  \item \verb|nDiscretize = 20| — 轨迹离散化点数
  \item \verb|nPaths = 20| — 每次迭代的采样数
  \item \verb|convergenceThreshold = 0.1| — 收敛阈值
  \item \verb|eta = 10| — Boltzmann 温度参数
\end{itemize}

\subsubsection{\texttt{stompSamples.m} — 采样生成}
为每个关节独立生成高斯噪声,使用 Cholesky 分解采样:
\begin{lstlisting}
A = chol(sigma, 'lower');
Z = randn(nDiscretize-2, nSamplePaths);
em_m = (A * Z)' + mu;  % (nPaths x innerN)
\end{lstlisting}
起点与终点不施加噪声(保持固定),仅对内部点 $t \in \{2,\dots,T-1\}$ 采样。

\subsubsection{\texttt{stompDTheta.m} — 梯度估计}
实现概率加权的噪声求和:
\begin{lstlisting}
dtheta = zeros(nJoints, nDiscretize_movable);
for m = 1:nJoints
    em_m = em{m};  % (nPaths x innerN)
    weighted_noise = trajProb .* em_m;  % Hadamard 积
    dtheta(m, :) = sum(weighted_noise, 1);  % 按列求和
end
\end{lstlisting}

\subsubsection{\texttt{stompObstacleCost.m} — 障碍代价}
关键实现细节:
\begin{itemize}
  \item 将球心坐标映射到体素网格索引
  \item 从 sEDT 提取符号距离 $s_i$
  \item 计算有效距离 $d_i = s_i - r_{\text{ball}}$
  \item 应用指数惩罚公式,仅对 $d_i < d_{\text{safe}}$ 的球施加代价
\end{itemize}

\subsubsection{\texttt{stompRobotSphere.m} — 碰撞球生成}
\textbf{关键优化:固定球数策略}

为避免相邻时间步球数不一致导致的维度不匹配错误,采用 \verb|persistent| 变量缓存每段连杆的球数量,确保整个规划过程中球总数恒定:
\begin{lstlisting}
persistent cachedCounts
if isempty(cachedCounts)
    for k = 1:nJoints
        L = norm(child_pos - parent_pos);
        cachedCounts(k) = max(2, ceil(L/rad) + 1);
    end
end
\end{lstlisting}

\subsection{实验设置与结果}

\subsubsection{实验环境}
\begin{itemize}
  \item 机器人:Kinova Gen3(7-DOF 机械臂)
  \item 工具箱:MATLAB Robotics System Toolbox
  \item 障碍物:由 \verb|helperCreateObstaclesKINOVA.m| 生成的 3D 体素环境
  \item 初末姿态:由逆运动学求解得到(\verb|taskInit|、\verb|taskFinal|)
\end{itemize}

\subsubsection{性能指标}
\begin{itemize}
  \item \textbf{碰撞检测}:使用 \verb|checkCollision| 验证最终轨迹无碰撞
  \item \textbf{代价收敛}:记录每轮迭代的总代价 $Q(\theta)$
  \item \textbf{平滑度}:计算控制代价 $\text{RAR} = \frac{1}{2}\theta^{\top}R\theta$
  \item \textbf{计算时间}:使用 \verb|tic/toc| 记录每次迭代耗时
\end{itemize}

\subsubsection{典型结果}
在默认参数设置下(\verb|nDiscretize=20|,\verb|nPaths=20|):
\begin{itemize}
  \item 算法在 \textbf{10-30 次迭代}内收敛(代价变化 $< 0.1$)
  \item 最终轨迹通过碰撞检测(\verb|isTrajectoryInCollision = false|)
  \item 障碍代价随迭代单调下降并趋近于零
  \item 平滑代价保持在合理范围,无明显关节抖动
  \item 单次迭代平均耗时约 \textbf{1-3 秒}(取决于硬件)
\end{itemize}

\subsection{讨论与改进}

\subsubsection{算法特性分析}
\begin{itemize}
  \item \textbf{优点}:无需梯度信息,适用于非光滑代价;并行化潜力大($K$ 条轨迹可独立评估);对初始化鲁棒。
  \item \textbf{局限}:对温度参数 $\eta$ 敏感;采样数 $K$ 较大时计算开销显著;可能陷入局部最优。
\end{itemize}

\subsubsection{参数调优经验}
\begin{itemize}
  \item 增大 \verb|nPaths| 可提高收敛稳定性,但需权衡计算时间
  \item 温度参数 \verb|eta=10| 在大多数场景表现良好;过小会使更新过于激进
  \item 安全裕度 $d_{\text{safe}}=0.1$m 需根据机器人尺寸与障碍密度调整
\end{itemize}

\subsubsection{潜在改进方向}
\begin{itemize}
  \item 采用\textbf{自适应温度}策略(迭代初期高温度鼓励探索,后期低温度精细收敛)
  \item 结合\textbf{多分辨率采样}(粗到细)加速收敛
  \item 集成\textbf{快速碰撞检测库}(如 FCL)替代 MATLAB 内置函数
\end{itemize}

\section{任务二}
% 按要求:仅保留标题,无正文

\section{任务三:基于 PoE 的正向运动学与 STOMP 集成}
\subsection{PoE 概述与螺旋轴确定}
PoE 模型以初始构型(零位)下的齐次变换 $M$ 和空间坐标系下的关节螺旋轴 $\{\mathbf{s}_i\}_{i=1}^n$ 描述串联机械臂。给定关节角 $\theta=[\theta_1,\dots,\theta_n]^\top$,末端位姿为:
\[
\mathbf{T}(\theta) = e^{\hat{\mathbf{s}}_1\theta_1}\,e^{\hat{\mathbf{s}}_2\theta_2}\,\cdots e^{\hat{\mathbf{s}}_n\theta_n}\,M\,.
\]
其中 $\hat{\mathbf{s}}$ 为 twist 的 4x4 反对称表示。螺旋轴可通过:\textit{(1)} 在零位读取各关节旋转轴与过轴点,用 $\omega, q$ 得到 $v=-\omega\times q$;\textit{(2)} 在非零位由关节相对变换取矩阵对数估计 twist 并归一化。README 提供了在 MATLAB 中提取参数的便捷方法。

\subsection{实现与替换 \texorpdfstring{\texttt{getTransform()}}{getTransform()}}
我们实现空间坐标系形式的 \verb|FKinSpace|(可参考 \cite{Lynch2017ModernRobotics} 的教材实现),并以其替代内置 \verb|getTransform()|。核心步骤:
\begin{enumerate}
  \item 预处理一次得到 $M$ 与 $\{\mathbf{s}_i\}$ 并缓存;
  \item 给定任意 \(\theta\) 时,按上式右乘推进得到末端(或任一连杆)位姿;
  \item 在 \verb|updateJointsWorldPosition| 中使用 PoE 统一计算世界系位姿,供 STOMP 的距离评估与可视化调用。
\end{enumerate}
这样避免了频繁调用内置函数的开销,且便于与 STOMP 的向量化评估整合。

\subsection{与 STOMP 的衔接与评估}
当 STOMP 对整条轨迹进行 $N$ 组采样时,PoE 允许我们以矩阵指数的链式形式快速批量计算 $\{\mathbf{T}_t^{(i)}\}$,配合向量化的障碍距离评估显著加速迭代。实验显示,在同等采样数下,使用 PoE 的实现可在保持精度的同时缩短评估时间(与具体硬件与 MATLAB 版本有关)。

\subsection{失败案例与改进}
若螺旋轴标定误差较大(如坐标系不一致),将导致位姿漂移或收敛失败。建议:统一参考系定义;对 $\theta$ 做角度归一化;在早期迭代提高平滑正则比重,待无碰撞后再降低以获得更短路径。

\section{任务四}
% 按要求:仅保留标题,无正文

\section{任务五}
% 按要求:仅保留标题,无正文

\section{结论与展望}
本文完成了 Task 1 与 Task 3:在原始障碍场景下以 STOMP 成功规划 Kinova Gen3 的无碰撞平滑轨迹,并以 PoE 实现替代了内置正向运动学,提升了可控性与运行效率。未来工作包括:更高维的约束建模(如姿态保持与力矩限制)、自适应采样与层级化优化、场景自动生成与评测基准完善等。

\begin{thebibliography}{9}
\bibitem{Kalakrishnan2011STOMP}
M. Kalakrishnan, S. Chitta, E. Theodorou, P. Pastor, and S. Schaal, ``STOMP: Stochastic Trajectory Optimization for Motion Planning,'' in \textit{IEEE International Conference on Robotics and Automation (ICRA)}, 2011.

\bibitem{Lynch2017ModernRobotics}
K. M. Lynch and F. C. Park, \textit{Modern Robotics: Mechanics, Planning, and Control}. Cambridge University Press, 2017. MATLAB 代码可参见:\url{https://github.com/NxRLab/ModernRobotics}。

\bibitem{Khatib1986Potential}
O. Khatib, ``Real-Time Obstacle Avoidance for Manipulators and Mobile Robots,'' \textit{The International Journal of Robotics Research}, 1986.

\end{thebibliography}

\end{document}


